\chapter{Generating Markov Chains With the PECOS Toolkit}\label{ch-gmc}
\thispagestyle{headings}
\markboth{Chapter \ref{ch-gmc}: Generating Markov Chains with the PECOS Toolkit}{Chapter \ref{ch-gmc}: Generating Markov Chains With the PECOS Toolkit}

Section \ref{sc-gmc-seven-steps} explains how to develop your own application using the DRAM capabilities of the PECOS Toolkit, while
Section \ref{sc-gmc-dram-examples} describes the examples available in the toolkit.
The chapter ends at Section \ref{sc-gmc-planned-features} with a brief list of planned features for next toolkit versions w.r.t. Markov Chain Monte Carlo methods.

\section{Seven Steps}\label{sc-gmc-seven-steps}

The process of developing an application (let us call it ``newexample'') 
that uses the DRAM implementation of the PECOS Toolkit involves the following seven steps:
\begin{enumerate}
\item prepare a file describing the {\it system input parameters}; let us call it ``newexample.par''; see Subsection \ref{subsc-gmc-seven-steps-sys-input-params};
\item create your {\it code directory}; let us call it ``uq/appls/mcmc/newexample''; See Subsection \ref{subsc-gmc-seven-steps-myexample};
\item if necessary, compute your {\it proposal covariance matrix} for the $q_1(\cdot,\cdot)$ proposal transition probability kernel; the PECOS library will internally compute a default covariance matrix if the user does not supply one; see Subsection \ref{subsc-gmc-seven-steps-proposal-cov-matrix-for-q1};
\item if necessary, code your {\it prior function}; there is a default prior in the PECOS library; see Subsection \ref{subsc-gmc-seven-steps-prior-code};
\item code your {\it likelihood function}, which returns a vector of values, each value corresponding to the likelihood of a specific output quantity, as explained in Section \ref{sc-intro-qoi}, page \pageref{sc-intro-qoi}; see Subsection \ref{subsc-gmc-seven-steps-likelihood-code};
\item {\it compile} your code; see Subsection \ref{subsc-gmc-seven-steps-compile};
\item prepare an input file setting the {\it algorithm parameters}; let us call it ``newexample.inp''; see Subsection \ref{subsc-gmc-seven-steps-alg-params}.
\end{enumerate}
Finally, {\it run} your application by executing ``./newexample -i newexample.inp''.

\subsection{File Describing System Input Parameters}\label{subsc-gmc-seven-steps-sys-input-params}

{\it All system input parameters are assumed to be scalar r.v.'s} (that is, $n_i=1,~i=0,1,\ldots,n_{\text{sip}}-1$, according to the terminology of Section \ref{sc-intro-qoi}), and
have to be specified in a file.
This file describing system input parameters contains a lot of prior information as well.
It is treated by the PECOS Toolkit as having either parameter lines or comment lines.
Comment lines are all those that begin with the character ``\#'', otherwise they are treated as parameter lines.
Comment lines can appear anywhere in the line.
Each parameter line is related to one input parameter.
The first parameter line specifies the input parameter $\theta_0$,
the second parameter line specifies the input parameter $\theta_1$,
and so on.
Let us ``$i$'' denote the index for the input parameter $\theta_i$.
From left to right, each parameter line shall contain the following fields separated by black space(s):
\begin{itemize}
\item name of $\theta_i$, without spaces: this field is mandatory;
\item $\theta_i^{(0)}$, the component value on the initial sample of the chain: this field is also mandatory;
\item $\theta_{i,\text{min}}$, the minimum value allowed for this component during chain generation, otherwise the candidate position is rejected: this field is optional and its default value is ``-inf'';
\item $\theta_{i,\text{max}}$, the maximum value allowed for this component during chain generation, otherwise the candidate position is rejected: this field is optional and its default value is ``inf'';
\item $E_{\text{prior}}[\theta_i]$, the expected value of the input parameter; this field is optional and the default value is ``nan'';
\item $V_{\text{prior}}[\theta_i]$, the variance of the input parameter; this field is optional and the default value is ``inf''.
\end{itemize}
If a parameter line is improperly set, the code will exit with a failure message.
An example of an input file is shown in Figure \ref{fig-dram-par-file-ex}.

\begin{figure}[h!]
\begin{verbatim}
# This is the file of system input parameters for "newexample".
# Tha name of this file must match the entry "uqParamSpace_inputFile" in
# the input file.
# Lines that begin with the character `#' are considered comment lines.
# Each line that is not a comment line is treated as representing an input
# parameter.
# From left to right, the entries in each input parameter line correspond to:
# --> parameter name (mandatory)
# --> initial value (mandatory)
# --> minimum value (optional; default value to "-inf")
# --> maximum value (optional; default value to "inf")
# --> expectation (optional; default value to "nan")
# --> variance (optional; default value to "inf")
Theta_0 0. -inf
# Comment lines can appear anywhere in the file.
Theta_1 0. -inf inf 0. inf
Theta_2 0.
Theta_3 0. -inf inf 0. inf
# The total number of input parameter lines (4 in this example) must match
# the entry "uqParamSpace_dim" in the input file.
\end{verbatim}
\caption{Example of the specification of system input parameters for the generation of a Markov chain by the PECOS Toolkit.
}
\label{fig-dram-par-file-ex}
\end{figure}

\subsection{Creating Your Code Directory}\label{subsc-gmc-seven-steps-myexample}

\begin{itemize}
\item execute ``uq/appls/mcmc/newexample'';
\item execute ``cp -R template newexample'';
\item execute ``cd newexample'';
\item change ``template'' and ``Template'' to ``newexample'' and ``newexample'' in all files, including the file ``Makefile''.
\end{itemize}

\subsection{The Proposal Covariance Matrix for $q_1(\cdot,\cdot)$}\label{subsc-gmc-seven-steps-proposal-cov-matrix-for-q1}
$~$\\

\subsection{Code for the Prior Fucntion}\label{subsc-gmc-seven-steps-prior-code}

The user might need to specify a prior probability density $\pi_{\text{prior}}(\boldsymbol{\theta})$
by passing to the PECOS Toolkit the address of a routine that computes
\begin{equation}\label{eq-m2l-prior}
-2~ln~
\left[
\pi_{\text{prior}}(\boldsymbol{\theta}).
\right]
\end{equation}
But the user specficication of a prior density is not mandatory, since the toolkit provides a default prior density given by
\begin{equation*}
\pi_{\text{prior,default}}(\boldsymbol{\theta}) =
e^{
\left\{
-\frac{1}{2}
(\boldsymbol{\theta}-E_{\text{prior}}[\boldsymbol{\theta}])^T
~C_{\text{prior}}^{-1}~
(\boldsymbol{\theta}-E_{\text{prior}}[\boldsymbol{\theta}])
\right\}
},
\end{equation*}
where the default covariance matrix $C_{\text{prior}}$ is the $n_{\text{sip}}\times n_{\text{sip}}$ diagonal matrix given by
\begin{equation}\label{eq-default-prior-cov-matrix}
C_{\text{prior}} =
\left[
\begin{array}{cccccc}
V_{\text{prior}}[\theta_0] & 0                          & 0                          & \ldots & 0                                           & 0      \\
0                          & V_{\text{prior}}[\theta_1] & 0                          & \ldots & 0                                           & 0      \\
0                          & 0                          & V_{\text{prior}}[\theta_2] & \ldots & 0                                           & 0      \\
\vdots                     & \vdots                     & \vdots                     & \ddots & \vdots                                      & \vdots \\
0                          & 0                          & 0                          & \ldots & V_{\text{prior}}[\theta_{n_{\text{sip}}-2}] & 0      \\
0                          & 0                          & 0                          & \ldots & 0                                           & V_{\text{prior}}[\theta_{n_{\text{sip}}-1}] \\
\end{array}
\right].
\end{equation}
In fact, in the spirit of \eqref{eq-m2l-prior}, the PECOS default prior routine computes
\begin{equation*}
(\boldsymbol{\theta}-E_{\text{prior}}[\boldsymbol{\theta}])^T
~C_{\text{prior}}^{-1}~
(\boldsymbol{\theta}-E_{\text{prior}}[\boldsymbol{\theta}]).
\end{equation*}
It should be noted that the values of $E_{\text{prior}}[\theta_i]$ and $V_{\text{prior}}[\theta_i]$, $i=0,1,\ldots,n_{\text{sip}}-1$, are passed in the file of system input parameters described in Subsection \ref{subsc-gmc-seven-steps-sys-input-params}.

\subsection{Code for the Likelihood Function}\label{subsc-gmc-seven-steps-likelihood-code}

The user must specify all likelihood densities $\ell_j(\mathbf{y}_{j,\text{obs}}|\boldsymbol{\theta})$, $j=0,1,\ldots,m_{\text{obs}}-1$,
by passing to the PECOS Toolkit the address of a routine that computes a vector $\mathbf{L}=(L_0,L_1,\ldots,L_{m_{\text{obs}}-1})\in\mathbb{R}^{m_{\text{obs}}}$  given by
\begin{equation}\label{eq-m2l-likelihood}
L_j = -2~ln~
\left[
\ell_j(\mathbf{y}_{j,\text{obs}}|\boldsymbol{\theta})
\right].
\end{equation}
The PECOS algorithm then computes
\begin{equation*}
\sum_{j=0}^{m_{\text{obs}}-1}~\frac{L_j}{V_{\text{prior}}[\ell_j(\cdot)]}
\end{equation*}

\subsection{Compiling Your Code}\label{subsc-gmc-seven-steps-compile}

Here you just need to execute ``make newexample''.

\subsection{File Describing Algorithm Parameters}\label{subsc-gmc-seven-steps-alg-params}

See Table \ref{tab-dram-map}.

\begin{sidewaystable}[h]
\begin{center}
\begin{tabular}{|l|c|c|c|c|}
\hline
\multicolumn{1}{|c|}{Option Name}        & Option Symbol or                  & \multicolumn{2}{c|}{Definition}                                                                 & Default Value \\
\cline{3-4}
                                         & Explanation                       & Equation or                                  & Page                                             &               \\
                                         &                                   & Section                                      &                                                  &               \\
\hline
\verb=uqParamSpace_dim=                  & $n_{\text{sip}}$                  & \eqref{eq-n-sip}                             & \pageref{eq-n-sip}                               &               \\
\hline
\verb=uqParamSpace_inputFile=            & A file name                       & \ref{subsc-gmc-seven-steps-sys-input-params} & \pageref{subsc-gmc-seven-steps-sys-input-params} &               \\
\hline
\verb=uqOutputSpace_dim=                 & $m_{\text{obs}}$                  & \eqref{eq-m-obs}                             & \pageref{eq-m-obs}                               &               \\
\hline
\verb=uqDRAM_mh_sizesOfChains=           & $n_{\text{chain}}$                & \eqref{eq-n-chain}                           & \pageref{eq-n-chain}                             &               \\
\hline
\verb=uqDRAM_dr_maxNumberOfExtraStages=  & $n_e$                             & \eqref{eq-ne}                                & \pageref{eq-ne}                                  &               \\
\hline
\verb=uqDRAM_dr_scalesForExtraStages=    & $\gamma_l$'s                      &                                              &                                                  &               \\
\hline
\verb=uqDRAM_am_initialNonAdaptInterval= & $t_0$                             & \eqref{eq-t0}                                & \pageref{eq-t0}                                  &               \\
\hline
\verb=uqDRAM_am_adaptInterval=           & $p_0$                             & \eqref{eq-p0}                                & \pageref{eq-p0}                                  &               \\
\hline
\verb=uqDRAM_am_eta=                     & $\eta$                            & \eqref{eq-eta}                               & \pageref{eq-eta}                                 &               \\
\hline
\verb=uqDRAM_am_epsilon=                 & $\epsilon$                        & \eqref{eq-epsilon}                           & \pageref{eq-epsilon}                             &               \\
\hline
\verb=uqDRAM_mh_lrSigma2Priors=          & $V_{\text{prior}}[\ell_j(\cdot)]$ &                                              &                                                  &               \\
\hline
\verb=uqDRAM_mh_lrUpdateSigma2=          & A boolean                         &                                              &                                                  &               \\
\hline
\verb=uqDRAM_mh_lrSigma2Accuracies=      & $a_j$'s                           &                                              &                                                  &               \\
\hline
\verb=uqDRAM_mh_lrNumbersOfObs=          & $m_j$'s                           &                                              &                                                  &               \\
\hline
\verb=uqDRAM_mh_namesOfOutputFiles=      & A file name                       &                                              &                                                  &               \\
\hline
\verb=uqDRAM_mh_chainDisplayPeriod=      & An integer                        &                                              &                                                  &               \\
\hline
\end{tabular}
\caption{Mapping between DRAM algorithm parameters in the input file of Figure \ref{fig-dram-input-file-ex} and the mathematical terms explained in Sections \ref{sc-intro-qoi} and \ref{sc-rmc-algs}.
}
\label{tab-dram-map}
\end{center}
\end{sidewaystable}

See Figure \ref{fig-dram-input-file-ex}.

\begin{figure}[h!]
\begin{verbatim}
###############################################
# UQ Parameter Space
###############################################
uqParamSpace_dim       = 4
uqParamSpace_inputFile = uqNormalEx.par

###############################################
# UQ Output Space
###############################################
uqOutputSpace_dim  = 1

###############################################
# UQ DRAM Markov Chain Generator
###############################################
uqDRAM_mh_sizesOfChains           = 5000
uqDRAM_mh_lrUpdateSigma2          = 0
uqDRAM_mh_lrSigma2Priors          = 1.
uqDRAM_mh_lrSigma2Accuracies      = 0.
uqDRAM_mh_lrNumbersOfObs          = 0
uqDRAM_dr_maxNumberOfExtraStages  = 0
uqDRAM_dr_scalesForExtraStages    = 5. 4. 3.
uqDRAM_am_initialNonAdaptInterval = 0
uqDRAM_am_adaptInterval           = 0
uqDRAM_am_sd                      = 1.44
uqDRAM_am_epsilon                 = 1.e-5
uqDRAM_mh_namesOfOutputFiles      = uqNormalExOutput.m
uqDRAM_mh_chainDisplayPeriod      = 500
\end{verbatim}
\caption{Example of an input file for the generation of a Markov Chain.
}
\label{fig-dram-input-file-ex}
\end{figure}

\section{Output Data}
$~$\\

\section{Examples Provided}\label{sc-gmc-dram-examples}

Three examples related to DRAM are provided: normal target distribution, chemical reactions and algae.
They are discussed in Subsections \ref{subsc-gmc-dram-normal-ex}, \ref{subsc-gmc-dram-chem-ex} and \ref{subsc-gmc-dram-algae-ex} respectively.

\subsection{Normal Target Distribution}\label{subsc-gmc-dram-normal-ex}

To be explained in future versions of the documentation.

See Table \ref{tab-dram-normal-ex-sys-input-params}.

\begin{table}[h!]
\begin{center}
\begin{tabular}{|c|c|c|c|c|c|c|}
\hline
 $i$      & Name of $\theta_i$ & $\theta_i^{(0)}$ & $\theta_{i,\text{min}}$ & $\theta_{i,\text{max}}$ & $E_{\text{prior}}[\theta_i]$ & $V_{\text{prior}}[\theta_i]$ \\
\hline
\hline
 $0$      & Param\_1           & $0.$             & $-\infty$               & $+\infty$               & $0.$                         & $+\infty$                    \\
\hline
 $1$      & Param\_2           & $0.$             & $-\infty$               & $+\infty$               & $0.$                         & $+\infty$                    \\
\hline
 $2$      & Param\_3           & $0.$             & $-\infty$               & $+\infty$               & $0.$                         & $+\infty$                    \\
\hline
 $3$      & Param\_4           & $0.$             & $-\infty$               & $+\infty$               & $0.$                         & $+\infty$                    \\
\hline
\end{tabular}
\caption{Normal target distribution example (Subsection \ref{subsc-gmc-dram-normal-ex}):
information on $n_{\text{sip}}=4$ system input parameters $\theta_i$, $0\leqslant i\leqslant n_{\text{sip}}-1$.
Each $\theta_i$ is a scalar r.v.. All system input parameters jointly form the vector r.v. $\boldsymbol{\Theta}=(\theta_0,\theta_1,\ldots,\theta_{n_{\text{sip}}-1})$.
}
\label{tab-dram-normal-ex-sys-input-params}
\end{center}
\end{table}

See Table \ref{tab-dram-normal-ex-alg-params}.

\begin{table}[h!]
\begin{center}
\begin{tabular}{|c|c|c|c|c|}
\hline
Option                                            & \multicolumn{4}{c|}{Algorithm}                               \\
\cline{2-5}
                                                  & ~~MH~~            & ~~DR~~       & ~~AM~~       & DRAM       \\
\hline
\hline
$n_{\text{sip}}$                                  &                   &              &              &            \\
\hline
$m_{\text{obs}}$                                  &                   &              &              &            \\
\hline
$n_{\text{chain}}$                                &                   &              &              &            \\
\hline
\hline
$n_e$                                             &                   &              &              &            \\
\hline
$\gamma_l$,
$0\leqslant l\leqslant n_e-1$                     &                   &              &              &            \\
\hline
\hline
$t_0$                                             &                   &              &              &            \\
\hline
$p_0$                                             &                   &              &              &            \\
\hline
$\eta$                                            &                   &              &              &            \\
\hline
$\epsilon$                                        &                   &              &              &            \\
\hline
\hline
$V_{\text{prior}}[\ell_j(\cdot)]$,
$0\leqslant j\leqslant m_{\text{obs}}-1$          & 1.                & 1.           & 1.           & 1.         \\
\hline
Update $V^{(k)}[\ell_j(\cdot)]$ for $k>0$?        & No                & No           & No           & No         \\
\hline
Values of $m_j$                                   & $-$               & $-$          & $-$          & $-$        \\
\hline
Values of $a_{\ell_j}$                            & $-$               & $-$          & $-$          & $-$        \\
\hline
\end{tabular}
\caption{Normal target distribution example (Subsection \ref{subsc-gmc-dram-normal-ex}):
algorithm parameters used.
}
\label{tab-dram-normal-ex-alg-params}
\end{center}
\end{table}

See Table \ref{tab-dram-normal-ex-results-1}.

\begin{table}[h!]
\begin{center}
\begin{tabular}{|c|c|c|c|c|c|c|c|c|}
\hline
Method & Run      & $\text{rej}$           & $\text{oor}$           & $i$ & $\langle\theta_i\rangle$ & $\sigma_{\text{bm}}(\theta_i)$ & $\text{gew}(\theta_i)$ & $\tau_{\text{int}}(\theta_i)$ \\
       & Time (s) &                        &                        &     &                          &                                &                        &                               \\
\hline
\hline
       &          &                        &                        & $0$ &                          &                                &                        &                               \\
\cline{5-9}
       &          &                        &                        & $1$ &                          &                                &                        &                               \\
\cline{5-9}
       &          &                        &                        & $2$ &                          &                                &                        &                               \\
\cline{5-9}
       &          &                        &                        & $3$ &                          &                                &                        &                               \\
\hline
\hline
       &          &                        &                        & $0$ &                          &                                &                        &                               \\
\cline{5-9}
       &          &                        &                        & $1$ &                          &                                &                        &                               \\
\cline{5-9}
       &          &                        &                        & $2$ &                          &                                &                        &                               \\
\cline{5-9}
       &          &                        &                        & $3$ &                          &                                &                        &                               \\
\hline
\hline
       &          &                        &                        & $0$ &                          &                                &                        &                               \\
\cline{5-9}
       &          &                        &                        & $1$ &                          &                                &                        &                               \\
\cline{5-9}
       &          &                        &                        & $2$ &                          &                                &                        &                               \\
\cline{5-9}
       &          &                        &                        & $3$ &                          &                                &                        &                               \\
\hline
\hline
       &          &                        &                        & $0$ &                          &                                &                        &                               \\
\cline{5-9}
       &          &                        &                        & $1$ &                          &                                &                        &                               \\
\cline{5-9}
       &          &                        &                        & $2$ &                          &                                &                        &                               \\
\cline{5-9}
       &          &                        &                        & $3$ &                          &                                &                        &                               \\
\hline
\end{tabular}
\caption{Normal target distribution example (Subsection \ref{subsc-gmc-dram-normal-ex}):
results from the first chain, with $n_{\text{chain}}=1001$ samples,
for each of the $n_{\text{sip}}=4$ parameters $\theta_i$, $0\leqslant i\leqslant n_{\text{sip}}-1$.
}
\label{tab-dram-normal-ex-results-1}
\end{center}
\end{table}

See Table \ref{tab-dram-normal-ex-results-2}.

\begin{table}[h!]
\begin{center}
\begin{tabular}{|c|c|c|c|c|c|c|c|c|}
\hline
Method & Run      & $\text{rej}$           & $\text{oor}$           & $i$ & $\langle\theta_i\rangle$ & $\sigma_{\text{bm}}(\theta_i)$ & $\text{gew}(\theta_i)$ & $\tau_{\text{int}}(\theta_i)$ \\
       & Time (s) &                        &                        &     &                          &                                &                        &                               \\
\hline
\hline
       &          &                        &                        & $0$ &                          &                                &                        &                               \\
\cline{5-9}
       &          &                        &                        & $1$ &                          &                                &                        &                               \\
\cline{5-9}
       &          &                        &                        & $2$ &                          &                                &                        &                               \\
\cline{5-9}
       &          &                        &                        & $3$ &                          &                                &                        &                               \\
\hline
\hline
       &          &                        &                        & $0$ &                          &                                &                        &                               \\
\cline{5-9}
       &          &                        &                        & $1$ &                          &                                &                        &                               \\
\cline{5-9}
       &          &                        &                        & $2$ &                          &                                &                        &                               \\
\cline{5-9}
       &          &                        &                        & $3$ &                          &                                &                        &                               \\
\hline
\hline
       &          &                        &                        & $0$ &                          &                                &                        &                               \\
\cline{5-9}
       &          &                        &                        & $1$ &                          &                                &                        &                               \\
\cline{5-9}
       &          &                        &                        & $2$ &                          &                                &                        &                               \\
\cline{5-9}
       &          &                        &                        & $3$ &                          &                                &                        &                               \\
\hline
\hline
       &          &                        &                        & $0$ &                          &                                &                        &                               \\
\cline{5-9}
       &          &                        &                        & $1$ &                          &                                &                        &                               \\
\cline{5-9}
       &          &                        &                        & $2$ &                          &                                &                        &                               \\
\cline{5-9}
       &          &                        &                        & $3$ &                          &                                &                        &                               \\
\hline
\end{tabular}
\caption{Normal target distribution example (Subsection \ref{subsc-gmc-dram-normal-ex}):
results from the second chain, with $n_{\text{chain}}=5001$ samples,
for each of the $n_{\text{sip}}=4$ parameters $\theta_i$, $0\leqslant i\leqslant n_{\text{sip}}-1$.
}
\label{tab-dram-normal-ex-results-2}
\end{center}
\end{table}

\subsection{Chemical Reactions}\label{subsc-gmc-dram-chem-ex}

To be explained in future versions of the documentation.

See Table \ref{tab-dram-chem-ex-sys-input-params}.

\begin{table}[h!]
\begin{center}
\begin{tabular}{|c|c|c|c|c|c|c|}
\hline
 $i$      & Name of $\theta_i$ & $\theta_i^{(0)}$ & $\theta_{i,\text{min}}$ & $\theta_{i,\text{max}}$ & $E_{\text{prior}}[\theta_i]$ & $V_{\text{prior}}[\theta_i]$ \\
\hline
\hline
 $0$      & Param\_1           & $0.$             & $-\infty$               & $+\infty$               & $0.$          & $+\infty$     \\
\hline
 $1$      & Param\_2           & $0.$             & $-\infty$               & $+\infty$               & $0.$          & $+\infty$     \\
\hline
 $2$      & Param\_3           & $0.$             & $-\infty$               & $+\infty$               & $0.$          & $+\infty$     \\
\hline
 $3$      & Param\_4           & $0.$             & $-\infty$               & $+\infty$               & $0.$          & $+\infty$     \\
\hline
\end{tabular}
\caption{Chemical reactions example (Subsection \ref{subsc-gmc-dram-chem-ex}):
information on $n_{\text{sip}}=4$ system input parameters $\theta_i$, $0\leqslant i\leqslant n_{\text{sip}}-1$.
Each $\theta_i$ is a scalar r.v.. All system input parameters jointly form the vector r.v. $\boldsymbol{\Theta}=(\theta_0,\theta_1,\ldots,\theta_{n_{\text{sip}}-1})$.
}
\label{tab-dram-chem-ex-sys-input-params}
\end{center}
\end{table}

See Table \ref{tab-dram-chem-ex-alg-params}.

\begin{table}[h!]
\begin{center}
\begin{tabular}{|c|c|c|c|c|}
\hline
Option                                            & \multicolumn{4}{c|}{Algorithm}                               \\
\cline{2-5}
                                                  & ~~MH~~            & ~~DR~~       & ~~AM~~       & DRAM       \\
\hline
\hline
$n_{\text{sip}}$                                  &                   &              &              &            \\
\hline
$m_{\text{obs}}$                                  &                   &              &              &            \\
\hline
$n_{\text{chain}}$                                &                   &              &              &            \\
\hline
\hline
$n_e$                                             &                   &              &              &            \\
\hline
$\gamma_l$,
$0\leqslant l\leqslant n_e-1$                     &                   &              &              &            \\
\hline
\hline
$t_0$                                             &                   &              &              &            \\
\hline
$p_0$                                             &                   &              &              &            \\
\hline
$\eta$                                            &                   &              &              &            \\
\hline
$\epsilon$                                        &                   &              &              &            \\
\hline
\hline
$V_{\text{prior}}[\ell_j(\cdot)]$,
$0\leqslant j\leqslant m_{\text{obs}}-1$          & 1.                & 1.           & 1.           & 1.         \\
\hline
Update $V^{(k)}[\ell_j(\cdot)]$ for $k>0$?        & No                & No           & No           & No         \\
\hline
Values of $m_j$                                   & $-$               & $-$          & $-$          & $-$        \\
\hline
Values of $a_{\ell_j}$                            & $-$               & $-$          & $-$          & $-$        \\
\hline
\end{tabular}
\caption{Chemical reactions example (Subsection \ref{subsc-gmc-dram-chem-ex}):
algorithm parameters used.
}
\label{tab-dram-chem-ex-alg-params}
\end{center}
\end{table}

See Table \ref{tab-dram-chem-ex-results-1}.

\begin{table}[h!]
\begin{center}
\begin{tabular}{|c|c|c|c|c|c|c|c|c|}
\hline
Method & Run      & $\text{rej}$           & $\text{oor}$           & $i$ & $\langle\theta_i\rangle$ & $\sigma_{\text{bm}}(\theta_i)$ & $\text{gew}(\theta_i)$ & $\tau_{\text{int}}(\theta_i)$ \\
       & Time (s) &                        &                        &     &                          &                                &                        &                               \\
\hline
\hline
       &          &                        &                        & $0$ &                          &                                &                        &                               \\
\cline{5-9}
       &          &                        &                        & $1$ &                          &                                &                        &                               \\
\cline{5-9}
       &          &                        &                        & $2$ &                          &                                &                        &                               \\
\cline{5-9}
       &          &                        &                        & $3$ &                          &                                &                        &                               \\
\hline
\hline
       &          &                        &                        & $0$ &                          &                                &                        &                               \\
\cline{5-9}
       &          &                        &                        & $1$ &                          &                                &                        &                               \\
\cline{5-9}
       &          &                        &                        & $2$ &                          &                                &                        &                               \\
\cline{5-9}
       &          &                        &                        & $3$ &                          &                                &                        &                               \\
\hline
\hline
       &          &                        &                        & $0$ &                          &                                &                        &                               \\
\cline{5-9}
       &          &                        &                        & $1$ &                          &                                &                        &                               \\
\cline{5-9}
       &          &                        &                        & $2$ &                          &                                &                        &                               \\
\cline{5-9}
       &          &                        &                        & $3$ &                          &                                &                        &                               \\
\hline
\hline
       &          &                        &                        & $0$ &                          &                                &                        &                               \\
\cline{5-9}
       &          &                        &                        & $1$ &                          &                                &                        &                               \\
\cline{5-9}
       &          &                        &                        & $2$ &                          &                                &                        &                               \\
\cline{5-9}
       &          &                        &                        & $3$ &                          &                                &                        &                               \\
\hline
\end{tabular}
\caption{Chemical reactions example (Subsection \ref{subsc-gmc-dram-chem-ex}):
results from the first chain, with $n_{\text{chain}}=1001$ samples,
for each of the $n_{\text{sip}}=4$ parameters $\theta_i$, $0\leqslant i\leqslant n_{\text{sip}}-1$.
}
\label{tab-dram-chem-ex-results-1}
\end{center}
\end{table}

See Table \ref{tab-dram-chem-ex-results-2}.

\begin{table}[h!]
\begin{center}
\begin{tabular}{|c|c|c|c|c|c|c|c|c|}
\hline
Method & Run      & $\text{rej}$           & $\text{oor}$           & $i$ & $\langle\theta_i\rangle$ & $\sigma_{\text{bm}}(\theta_i)$ & $\text{gew}(\theta_i)$ & $\tau_{\text{int}}(\theta_i)$ \\
       & Time (s) &                        &                        &     &                          &                                &                        &                               \\
\hline
\hline
       &          &                        &                        & $0$ &                          &                                &                        &                               \\
\cline{5-9}
       &          &                        &                        & $1$ &                          &                                &                        &                               \\
\cline{5-9}
       &          &                        &                        & $2$ &                          &                                &                        &                               \\
\cline{5-9}
       &          &                        &                        & $3$ &                          &                                &                        &                               \\
\hline
\hline
       &          &                        &                        & $0$ &                          &                                &                        &                               \\
\cline{5-9}
       &          &                        &                        & $1$ &                          &                                &                        &                               \\
\cline{5-9}
       &          &                        &                        & $2$ &                          &                                &                        &                               \\
\cline{5-9}
       &          &                        &                        & $3$ &                          &                                &                        &                               \\
\hline
\hline
       &          &                        &                        & $0$ &                          &                                &                        &                               \\
\cline{5-9}
       &          &                        &                        & $1$ &                          &                                &                        &                               \\
\cline{5-9}
       &          &                        &                        & $2$ &                          &                                &                        &                               \\
\cline{5-9}
       &          &                        &                        & $3$ &                          &                                &                        &                               \\
\hline
\hline
       &          &                        &                        & $0$ &                          &                                &                        &                               \\
\cline{5-9}
       &          &                        &                        & $1$ &                          &                                &                        &                               \\
\cline{5-9}
       &          &                        &                        & $2$ &                          &                                &                        &                               \\
\cline{5-9}
       &          &                        &                        & $3$ &                          &                                &                        &                               \\
\hline
\end{tabular}
\caption{Chemical reactions example (Subsection \ref{subsc-gmc-dram-chem-ex}):
results from the second chain, with $n_{\text{chain}}=5001$ samples,
for each of the $n_{\text{sip}}=4$ parameters $\theta_i$, $0\leqslant i\leqslant n_{\text{sip}}-1$.
}
\label{tab-dram-chem-ex-results-2}
\end{center}
\end{table}

\subsection{Algae}\label{subsc-gmc-dram-algae-ex}

\section{Planned Features for Next Releases}\label{sc-gmc-planned-features}
With respect to Markov Chain Monte Carlo methods, the following features are planned for the next versions of the PECOS Toolkit for Predictive Engineering:
\begin{enumerate}
\item compute prediction with the chain of the algae example,
\item capability of running over parallel environments using Trilinos,
\item integration with the DAKOTA Toolkit,
\item chain convergence tests \cite{BrRo98},
\item capability of running over parallel environments using PETSc,
\item hyperprior models,
\item algorithms for multimodal distributions,
\item Gibbs sampler.
\end{enumerate}
 
