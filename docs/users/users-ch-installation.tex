\chapter{Installation}\label{ch-install}
\thispagestyle{headings}
\markboth{Chapter \ref{ch-install}: Installation}{Chapter \ref{ch-install}: Installation}

The PECOS Toolkit currently runs over uniprocessor Linux environments and requires the following packages to be installed in your computing system:
\begin{itemize}
\item GNU Scientific Library (gsl) \cite{gsl},
\item Boost C++ Libraries \cite{boost} and
\item MCMC Toolbox for Matlab \cite{mcmctool}.
\end{itemize}
The gsl library is used for many mathematical operations.
The ``Program Options'' library from the Boost C++ Libraries is necessary for the reading of parameters from input files.
The MCMC Toolbox for Matlab is necessary for the plotting of some of the data generated by the PECOS Toolkit.

\section{Installation Steps}

In order to install the PECOS Toolkit, you have to:
\begin{itemize}
\item {download it:
\begin{itemize}
\item mkdir $<$WORK\_DIR$>$
\item download ``pecos\_toolkit\_v\_0\_1.tar.gz'' from http://www.ices.utexas.edu/centers/pecos to $<$WORK\_DIR$>$
\item cd $<$WORK\_DIR$>$
\item gunzip pecos\_toolkit\_v\_0\_1.tar.gz
\item tar -xvf pecos\_toolkit\_v\_0\_1.tar
\item gzip pecos\_toolkit\_v\_0\_1.tar
\end{itemize}
}
\item {prepare the computational environment:
\begin{itemize}
\item set the variable BOOST\_INCLUDE\_PATH\_FOR\_PECOS\_TOOLKIT
\item set the variable BOOST\_LIB\_PATH\_FOR\_PECOS\_TOOLKIT
\item set the variable MCMC\_TOOLBOX\_PATH\_FOR\_PECOS\_TOOLKIT
\item add ``\$BOOST\_LIB\_PATH\_FOR\_PECOS\_TOOLKIT'' to LD\_LIBRARY\_PATH
\end{itemize}
}
\item {compile the code:
\begin{itemize}
\item cd $<$WORK\_DIR$>$/uq
\item make clean\_all UQBT=gsl
\item make UQBT=gsl
\end{itemize}
}
\end{itemize}
The PECOS Toolkit assumes that the MCMC Toolbox for Matlab will have:
\begin{itemize}
\item its core code located at \$MCMC\_TOOLBOX\_PATH\_FOR\_PECOS\_TOOLKIT/code, and
\item its examples located at \$MCMC\_TOOLBOX\_PATH\_FOR\_PECOS\_TOOLKIT/examples.
\end{itemize}
Also, if you wish to run the Sandia's DAKOTA Toolkit \cite{dakota} with the applications provided by the PECOS Toolkit, then you will need to:
\begin{itemize}
\item add ``$<$WORK\_DIR$>$/uq/appls/fp/heatex'' to your PATH.
\end{itemize}

If you want to generate documentation, just run
\begin{itemize}
\item cd $<$WORK\_DIR$>$/uq
\item make all\_docs
\end{itemize}
These commands will generate:
\begin{itemize}
\item the User's Manual in pdf format at $<$WORK\_DIR$>$/uq/docs/users,
\item the Developer's Manual in pdf format at $<$WORK\_DIR$>$/uq/docs/developers, and
\item ./html/ subdirectories containing lower-level code documentation throughout the PECOS toolkit directory.
\end{itemize}
The user is able to navigate through the lower-level code documentation by pointing a web browser to $<$WORK\_DIR$>$/uq/html/index.html.

\section{Directory Tree Overview}

Under the directory ``$<$WORK\_DIR$>$/uq/'' there are three subdirectories:
\begin{itemize}
\item uq/docs/, which contains the User's Manual and the Developer's Manual,
\item {uq/appls/, which contains examples of applications for
\begin{itemize}
\item MCMC methods (uq/appls/mcmc/) and
\item forward propagation computations using DAKOTA (uq/appls/fp/), and
\end{itemize}
}
\item uq/libs/, which contains the libraries used by the applications under uq/appls/.
\end{itemize}

The examples for MCMC methods are explained in Setion \ref{sc-gmc-dram-examples}.
The examples for forward propagation are explained in Section \ref{sc-fp-examples}.
The libraries under uq/libs are explained in the Developer's Manual.
