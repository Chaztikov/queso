%------------------------------------------------------------------
\thispagestyle{empty}
{\setlength{\parindent}{0cm}\bf{PECOS Toolkit for Predictive Engineering}}\hfill $~$\\
\begin{picture}(8,0.1)
\linethickness{3pt}
\put(0,0.1){\line(1,0){6.6}}
\end{picture}
$~$\hfill User's Manual\\
$~$\hfill Version 0.1\\
$~$\hfill July 17, 2008\\

\vfill
$~$\\
{\bf{Ernesto E. Prudencio}}\hfill\\
{\bf{Chris Simmons}}\hfill\\
{\bf{Omar Ghattas}}\hfill\\
{\bf{Robert Moser}}\hfill\\
Center for Predictive Engineering and Computational Sciences (PECOS) \hfill\\
Institute for Computational and Engineering Sciences (ICES) \hfill\\
University of Texas at Austin\hfill\\

\vfill
$~$\\
\begin{picture}(8,0.1)
\linethickness{1.5pt}
\put(0,0.1){\line(1,0){6.6}}
\end{picture}

\clearpage
%------------------------------------------------------------------
\thispagestyle{empty}
$~$\\
\vfill
Copyright \copyright\ 2008 The PECOS Team, \texttt{http://www.ices.utexas.edu/centers/pecos.}\\
Permission is granted to copy, distribute and/or modify this document under the terms of
the GNU Free Documentation License, Version 1.2 or any later version published by the Free
Software Foundation; with the Invariant Sections being ``GNU General Public License'' and
``Free Software Needs Free Documentation'', the Front-Cover text being ``A GNU Manual'',
and with the Back-Cover text being ``You have the freedom to copy and modify this GNU Manual''.
A copy of the license is included in the section entitled ``GNU Free Documentation License''.

\clearpage
%------------------------------------------------------------------
\addcontentsline{toc}{chapter}{Abstract}
%\thispagestyle{empty}
\centerline{\Large\bf Abstract}
$~$\\
The PECOS Toolkit for Predictive Engineering is a
collection of algorithms, libraries and executables aimed for
the validation of a model and
the prediction of quantities of interest from such model
along with the quantification of their uncertainties.
It is designed for flexibility, portability, easiness of use and easiness of extension.
Its software design follows an object-oriented approach and
its code is written on C++ and over MPI.
It can run over uniprocessor or multiprocessor environments.

The PECOS Toolkit contains three forms of documentation:
a User's Manual available in pdf format,
a Developer's Manual also available in pdf format, and
a lower-level code documentation available in web based/html format.

This is the User's Manual.
It gives an overview of the toolkit capabilities,
provides procedures for software execution, and includes example studies.

\clearpage
%------------------------------------------------------------------
$~$\\

\clearpage
%------------------------------------------------------------------
\addcontentsline{toc}{chapter}{Disclaimer}
%\thispagestyle{empty}
\centerline{\Large\bf Disclaimer}
$~$\\
    THE PECOS TOOLKIT FOR PREDICTIVE ENGINEERING WAS PREPARED AS AN ACCOUNT OF A JOINT WORK
    BETWEEN THE UNIVERSITY OF TEXAS AT AUSTIN AND AN AGENCY OF THE UNITED STATES GOVERNMENT.
    NEITHER THE UNITED STATES GOVERNMENT NOR ANY AGENCY THEREOF, NOR THE UNIVERSITY OF TEXAS
    AT AUSTIN, NOR ANY OF THEIR EMPLOYEES OR OFFICERS, MAKES ANY WARRANTY, EXPRESS OR IMPLIED,
    OR ASSUMES ANY LEGAL LIABILITY OR RESPONSIBILITY FOR THE ACCURACY, COMPLETENESS, OR
    USEFULNESS OF ANY INFORMATION, APPARATUS, PRODUCT, OR PROCESS DISCLOSED, OR REPRESENTS
    THAT ITS USE WOULD NOT INFRINGE PRIVATELY OWNED RIGHTS. REFERENCE HEREIN TO ANY SPECIFIC
    COMMERCIAL PRODUCT, PROCESS, OR SERVICE BY TRADE NAME, TRADEMARK, MANUFACTURER, OR OTHERWISE,
    DOES NOT NECESSARILY CONSTITUTE OR IMPLY ITS ENDORSEMENT, RECOMMENDATION, OR FAVORING BY
    THE UNIVERSITY OF TEXAS AT AUSTIN, THE UNITED STATES GOVERNMENT OR ANY AGENCY THEREOF.
    THE VIEW AND OPINIONS OF AUTHORS EXPRESSED HEREIN DO NOT NECESSARILY STATE OR REFLECT
    THOSE OF THE UNIVERSITY OF TEXAS AT AUSTIN, THE UNITED STATES GOVERNMENT OR ANY AGENCY
    THEREOF.

\clearpage
%------------------------------------------------------------------
$~$\\

\clearpage
%------------------------------------------------------------------
{\markboth{}{}
\addtocontents{toc}{\protect\markboth{}{}}
}
%\addtocontents{toc}{\protect\thispagestyle{headings}}
\tableofcontents

\clearpage
%------------------------------------------------------------------
$~$\\

\clearpage
%------------------------------------------------------------------
\addcontentsline{toc}{chapter}{Preface}
\thispagestyle{empty}
\centerline{\Large\bf Preface}
$~$\\
The PECOS (Predictive Engineering and Computational Sciences) Toolkit project started in 2008 as part
of the efforts of the recently established Center for Predictive Engineering and Computational Sciences (PECOS)
by the Institute for Computational and Engineering Sciences (ICES) at the University of Texas at Austin.
The PECOS Center was selected by the National Nuclear Security Administration (NNSA) as one of its new five centers of excellence
under the Predictive Science Academic Alliance Program (PSAAP).

The goal of the PECOS Center is
to advance predictive science and to develop the next generation of advanced computational methods and tools
for the calculation of reliable predictions on the behavior of complex phenomena and systems (multiscale, multidisciplinary).
This objective demands a systematic, comprehensive treatment of the calibration and validation of the mathematical models involved,
as well as the quantification of the uncertainties inherent in such models.
The advancement of predictive science is essential for the application of Computational Science to the solution of realistic problems of national interest.

The PECOS Toolkit, since its first version, has been publicly released as open source
under the GNU General Public License and is available for free download world-wide.
See http://www.gnu.org/licenses/gpl.html for more information on the GPL software use agreement.

The PECOS Toolkit team consists of
Ernesto E. Prudencio (Research Associate),
Chris Simmons (Program Director),
Omar Ghattas (Co-PI) and
Robert Moser (PI and Director).

{\bf Contact Information:}\\
Ernesto E. Prudencio\\
Institute for Computational and Engineering Sciences\\
1 University Station C0200\\
Austin, Texas 78712

email: prudenci@ices.utexas.edu\\
web: http://www.ices.utexas.edu/centers/pecos\\
$~$\\

\centerline{\bf Referencing the PECOS Toolkit}

When referencing the PECOS Toolkit for Predictive Engineering in a publication, please cite the following:
\begin{verbatim}
@Misc{pecos-web-page,
   Author = ``Omar Ghattas and Robert Moser and
              Ernesto E. Prudencio and Chris Simmons'',
   Title      = ``{C}enter for {P}redictive {E}ngineering and
                  {C}omputational {S}ciences ({PECOS}) {W}eb page'',
   Note     = ``http://www.ices.utexas.edu/pecos'',
   Year     = ``2008''}

@TechReport{pecos-user-ref,
   Author      = ``Omar Ghattas and Robert Moser and
                   Ernesto E. Prudencio and Chris Simmons'',
   Title       = ``{PECOS} {T}oolkit for {P}redictive {E}ngineering
                   {U}ser's {M}anual'',
   Institution = ``Institute for Computational and Engineering
                   Sciences (ICES), University of Texas at Austin'',
   Year        = ``2008''}
\end{verbatim}
$~$\\
$~$\\

\centerline{\bf Acknowledgements}

%\clearpage
%%------------------------------------------------------------------
%$~$\\
