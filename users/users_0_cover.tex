	%------------------------------------------------------------------
\thispagestyle{empty}
{\setlength{\parindent}{0cm}\bf{The QUESO Library}}\hfill $~$\\
\begin{picture}(8,0.1)
\linethickness{3pt}
\put(0,0.1){\line(1,0){6.6}}
\end{picture}
$~$\hfill User's Manual\\
$~$\hfill Version 0.41.0\\
$~$\hfill October 30, 2009\\

\vfill
$~$\\
\begin{center}
{\large\bf Quantification of Uncertainty for Estimation,}\\
{\large\bf Simulation, and Optimization (QUESO)}\\
\end{center}
$~$\\

%\vfill
%$~$\\
%{\bf TERMINOLOGY USED IN THIS MANUAL IS SUBJECT TO CHANGE}

\vfill
%$~$\\
%{\bf Lead Developer:}\hfill \\
%$~\hspace{10pt}$ {\em{Ernesto E. Prudencio}}\hfill\\ 
$~$\\
{\bf Contributors:}\hfill \\
$~\hspace{10pt}$ {\em{Paul T. Bauman}}  \hfill \\
$~\hspace{10pt}$ {\em{Sai Hung Cheung}} \hfill \\
$~\hspace{10pt}$ {\em{Todd A. Oliver}}  \hfill \\
$~\hspace{10pt}$ {\em{Ernesto E. Prudencio}} \hfill\\ 
$~\hspace{10pt}$ {\em{Karl W. Schulz}}  \hfill \\
$~\hspace{10pt}$ {\em{Rhys Ulerich}}    \hfill \\

\vfill
$~$\\
\begin{center}
Center for Predictive Engineering and Computational Sciences (PECOS) \hfill\\
Institute for Computational and Engineering Sciences (ICES) \hfill\\
The University of Texas at Austin\hfill\\
\end{center}

\vfill
$~$\\
\begin{picture}(8,0.1)
\linethickness{1.5pt}
\put(0,0.1){\line(1,0){6.6}}
\end{picture}

\clearpage
%------------------------------------------------------------------
\thispagestyle{empty}
$~$\\
\vfill
Copyright \copyright\ 2008-2009 The PECOS Development Team, \texttt{http://pecos.ices.utexas.edu}\\
Permission is granted to copy, distribute and/or modify this document under the terms of
the GNU Free Documentation License, Version 1.2 or any later version published by the Free
Software Foundation; with the Invariant Sections being ``GNU General Public License'' and
``Free Software Needs Free Documentation'', the Front-Cover text being ``A GNU Manual'',
and with the Back-Cover text being ``You have the freedom to copy and modify this GNU Manual''.
A copy of the license is included in the section entitled ``GNU Free Documentation License''.

\clearpage
%------------------------------------------------------------------
\addcontentsline{toc}{chapter}{Abstract}
%\thispagestyle{empty}
\centerline{\Large\bf Abstract}
$~$\\
QUESO is a collection of algorithms and C++ classes aimed for
research in uncertainty quantification,
including
the solution of statistical inverse and statistical forward problems,
the validation of mathematical models under uncertainty and
the prediction of quantities of interest from such models along with
the quantification of their uncertainties.

QUESO is designed for flexibility, portability, easiness of use and
easiness of extension. Its software design follows an object-oriented
approach and its code is written on C++ and over MPI. It can run over
uniprocessor or multiprocessor environments.

QUESO contains two forms of documentation:
a User's Manual available in pdf format
and
a lower-level code documentation available in web based/html format.

This is the User's Manual.
It gives an overview of the QUESO capabilities,
provides procedures for software execution, and includes example studies.

\clearpage
%------------------------------------------------------------------
$~$\\

\clearpage
%------------------------------------------------------------------
\addcontentsline{toc}{chapter}{Disclaimer}
%\thispagestyle{empty}
\centerline{\Large\bf Disclaimer (To be checked by Karl)}
$~$\\
    THIS DOCUMENT WAS PREPARED
    BY THE UNIVERSITY OF TEXAS AT AUSTIN.
    NEITHER THE UNIVERSITY OF TEXAS
    AT AUSTIN, NOR ANY OF ITS INSTITUTES, DEPARTMENTS AND EMPLOYEES, MAKES ANY WARRANTY, EXPRESS OR IMPLIED,
    OR ASSUMES ANY LEGAL LIABILITY OR RESPONSIBILITY FOR THE ACCURACY, COMPLETENESS, OR
    USEFULNESS OF ANY INFORMATION, APPARATUS, PRODUCT, OR PROCESS DISCLOSED, OR REPRESENTS
    THAT ITS USE WOULD NOT INFRINGE PRIVATELY OWNED RIGHTS. REFERENCE HEREIN TO ANY SPECIFIC
    COMMERCIAL PRODUCT, PROCESS, OR SERVICE BY TRADE NAME, TRADEMARK, MANUFACTURER, OR OTHERWISE,
    DOES NOT NECESSARILY CONSTITUTE OR IMPLY ITS ENDORSEMENT, RECOMMENDATION, OR FAVORING BY
    THE UNIVERSITY OF TEXAS AT AUSTIN OR ANY OF ITS INSTITUTES, DEPARTMENTS AND EMPLOYEES THEREOF.
    THE VIEW AND OPINIONS EXPRESSED HEREIN DO NOT NECESSARILY STATE OR REFLECT
    THOSE OF THE UNIVERSITY OF TEXAS AT AUSTIN OR ANY INSTITUTE OR DEPARTMENT
    THEREOF.

\clearpage
%------------------------------------------------------------------
$~$\\

\clearpage
%------------------------------------------------------------------
{\markboth{}{}
\addtocontents{toc}{\protect\markboth{}{}}
}
%\addtocontents{toc}{\protect\thispagestyle{headings}}
\tableofcontents

%\clearpage
%%------------------------------------------------------------------
%$~$\\

\clearpage
%------------------------------------------------------------------
\addcontentsline{toc}{chapter}{Preface}
\thispagestyle{empty}
\centerline{\Large\bf Preface}
$~$\\
The QUESO project started in 2008 as part
of the efforts of the recently established Center for Predictive Engineering and Computational Sciences (PECOS)
at the Institute for Computational and Engineering Sciences (ICES) at The University of Texas at Austin.

The PECOS Center was selected by the National Nuclear Security Administration (NNSA) as one of its new five centers of excellence
under the Predictive Science Academic Alliance Program (PSAAP).
The goal of the PECOS Center is
to advance predictive science and to develop the next generation of advanced computational methods and tools
for the calculation of reliable predictions on the behavior of complex phenomena and systems (multiscale, multidisciplinary).
This objective demands a systematic, comprehensive treatment of the calibration and validation of the mathematical models involved,
as well as the quantification of the uncertainties inherent in such models.
The advancement of predictive science is essential for the application of Computational Science to the solution of realistic problems of national interest.

The QUESO library, since its first version, has been publicly released as open source
under the GNU General Public License and is available for free download world-wide.
See http://www.gnu.org/licenses/gpl.html for more information on the GPL software use agreement.

The QUESO development team currently consists of
Paul T. Bauman,
Sai Hung Cheung,
Todd A. Oliver,
Ernesto E. Prudencio,
Karl W. Schulz, and
Rhys Ulerich.

{\bf Contact Information:}\\
Ernesto E. Prudencio\\
Institute for Computational and Engineering Sciences\\
1 University Station C0200\\
Austin, Texas 78712

email: prudenci@ices.utexas.edu\\
web: http://pecos.ices.utexas.edu\\
$~$\\

\centerline{\bf Referencing the QUESO Library}

When referencing the QUESO library in a publication, please cite the following:
\begin{verbatim}
@Misc{queso-web-page,
   Author = "Ernesto E. Prudencio and Paul T. Bauman and Sai Hung Cheung
             and Todd A. Oliver and Karl W. Schulz and Rhys Ulerich",
   Title  = "{T}he {QUESO} {L}ibrary: {Q}uantification of {U}ncertainty
             for {E}stimation, {S}imulation and {O}ptimization",
   Note   = "http://pecos.ices.utexas.edu",
   Year   = "2008-2009"}

@TechReport{queso-user-ref,
   Author      = "Ernesto E. Prudencio and Paul T. Bauman and Sai Hung Cheung
                  and Todd A. Oliver and Karl W. Schulz and Rhys Ulerich",
   Title       = "{T}he {QUESO} {L}ibrary, {U}ser's {M}anual, {ICES} {T}echnical
                  {R}eport XXYYZZ",
   Institution = "Center for Predictive Engineering and Computational Sciences
                  (PECOS), at the Institute for Computational and Engineering
                  Sciences (ICES), The University of Texas at Austin",
   Year        = "2008-2009"}
\end{verbatim}
$~$\\
$~$\\

\centerline{\bf Acknowledgments}

Portions of this material is based upon work supported by the Department of Energy [National Nuclear Security Administration] under Award Number [DE-FC52-08NA28615].

We would also like to thank
James Martin,
Roy Stogner and
Lucas Wilcox
for interesting discussions and constructive feedbacks.

%\clearpage
%%------------------------------------------------------------------
%$~$\\
