\chapter{C++ Classes in the Library}\label{ch-classes}
\thispagestyle{headings}
\markboth{Chapter \ref{ch-classes}: C++ Classes in the Library}{Chapter \ref{ch-classes}: C++ Classes in the Library}

%The QUESO-MCMC Tool currently implements the DRAM algorithm \cite{HaLaMiSa06} for the generation of a Markov chain.
%Section \ref{sc-gmc-eight-steps} explains how to develop your own application using the DRAM capabilities of the QUESO-MCMC Tool, while
%Section \ref{sc-gmc-dram-output} describes the output information generated by the toolkit and
%Sections
%\ref{sc-gmc-dram-normal-ex},
%\ref{sc-gmc-dram-chem-ex} and
%\ref{sc-gmc-dram-algae-ex}
%describe the three examples available,
%all of them also available in \cite{mcmctool}.

%The chapter ends at Section \ref{sc-gmc-planned-features} with a brief list of planned features for next toolkit versions w.r.t. Markov Chain Monte Carlo methods.

\section{Core Classes}

The are four core classes:
\begin{itemize}
\item environment and environment options (Figures \ref{fig-env-class} and \ref{fig-env-options-class}, pages \pageref{fig-env-class} and \pageref{fig-env-options-class}),
\item vector (Figure \ref{fig-vector-class}, page \pageref{fig-vector-class}),
\item matrix (Figure \ref{fig-matrix-class}, page \pageref{fig-matrix-class}).
\end{itemize}

\clearpage
\subsection{Environment (and Options)}

\begin{figure}[h!]
\begin{center}
\includegraphics[scale=0.40,clip=true]{figs/uqEnvironment.eps}
%\includegraphics[scale=0.55,clip=true]{figs/uqEnvironment.eps}
\end{center}
\caption{
The class diagram for the environment class.
}
\label{fig-env-class}
\end{figure}

\begin{figure}[h!]
\begin{center}
\includegraphics[scale=0.40,clip=true]{figs/uqEnvironmentOptions.eps}
\end{center}
\caption{
The environment options class.
}
\label{fig-env-options-class}
\end{figure}

\begin{table}[!h]
\begin{center}
\begin{tabular}{|l|c|c|}
\hline
\multicolumn{1}{|c|}{Option}               & Default & Description \\
\multicolumn{1}{|c|}{Name}                 & Value   &             \\
\hline
\hline
$\langle$PREFIX$\rangle$help               &         &             \\
\hline
$\langle$PREFIX$\rangle$numSubEnvironments &         &             \\
\hline
$\langle$PREFIX$\rangle$subDisplayFileName &         &             \\
\hline
$\langle$PREFIX$\rangle$subDisplayAllowAll &         &             \\
\hline
$\langle$PREFIX$\rangle$subDisplayAllow    &         &             \\
\hline
$\langle$PREFIX$\rangle$displayVerbosity   &         &             \\
\hline
$\langle$PREFIX$\rangle$syncVerbosity      &         &             \\
\hline
$\langle$PREFIX$\rangle$seed               &         &             \\
\hline
\end{tabular}
\end{center}
\caption{
Input file options for a QUESO environment.
}
\label{tab-env-options}
\end{table}

\clearpage
\subsection{Vector}

\begin{figure}[h!]
\centerline{
\includegraphics[scale=0.40,clip=true]{figs/uqVector.eps}
}
\caption{
The class diagram for the vector class.
}
\label{fig-vector-class}
\end{figure}

\clearpage
\subsection{Matrix}

\begin{figure}[h!]
\centerline{
\includegraphics[scale=0.40,clip=true]{figs/uqMatrix.eps}
}
\caption{
The class diagram for the matrix class.
}
\label{fig-matrix-class}
\end{figure}

\clearpage
\section{Miscellaneous Classes and Routines}

\clearpage
\section{Templated Basic Classes}

The classes in this group are:
\begin{itemize}
\item Vector sets, subsets and spaces (see Figure \ref{fig-vector-space-subset-classes}),
\item Scalar function (see Figure \ref{fig-scalar-function-class}),
\item Vector function (see Figure \ref{fig-vector-function-class}),
\item Scalar sequence (see Figure \ref{fig-scalar-sequence-class}), and
\item Vector sequence (see Figure \ref{fig-vector-sequence-class}).
\end{itemize}
These classes constitute the core entities necessary for the formal
mathematical definition and description of other entities, such as
random variables, Bayesian solutions of inverse problems, sampling algorithms and chains.

\clearpage
\subsection{Vector Subset and Vector Space}

\begin{figure}[h!]
\centerline{
\includegraphics[scale=0.40,clip=true]{figs/uqVectorSet.eps}
}
\caption{
The class diagram for vector set, vector subset and vector space classes.
}
\label{fig-vector-space-subset-classes}
\end{figure}

\clearpage
\subsection{Scalar Function}

\begin{figure}[h!]
\centerline{
\includegraphics[scale=0.40,clip=true]{figs/uqScalarFunction.eps}
}
\caption{
The class diagram for the scalar function class.
}
\label{fig-scalar-function-class}
\end{figure}

\clearpage
\subsection{Vector Function}

\begin{figure}[h!]
\centerline{
\includegraphics[scale=0.40,clip=true]{figs/uqVectorFunction.eps}
}
\caption{
The class diagram for the vector function class.
}
\label{fig-vector-function-class}
\end{figure}

\clearpage
\subsection{Scalar Sequence}

\begin{figure}[h!]
\centerline{
\includegraphics[scale=0.40,clip=true]{figs/uqScalarSequence.eps}
}
\caption{
The class diagram for the scalar sequence class.
}
\label{fig-scalar-sequence-class}
\end{figure}

\clearpage
\subsection{Vector Sequence}

\begin{figure}[h!]
\centerline{
\includegraphics[scale=0.40,clip=true]{figs/uqVectorSequence.eps}
}
\caption{
The class diagram for the vector sequence class.
}
\label{fig-vector-sequence-class}
\end{figure}

\clearpage
\section{Templated Statistical Classes}

\begin{itemize}
\item Vector random variable
\item Statistical inverse problem (and options)
\item Markov chain solver (and options)
\item Statistical forward problem (and options)
\item Monte Carlo solver (and options)
\item Sequence statistical options
\end{itemize}

For QUESO, a statistical inverse problem has two input entities, a prior random variable and
a likelihood routine, and one output entity, the posterior random variable, as shown in Figure \ref{fig-sip-queso}.

Similarly, a statistical forward problem for QUESO has two input entities, a input random variable and
a qoi routine, and one output entity, the output random variable, as shown in Figure \ref{fig-sfp-queso}.

\clearpage
\subsection{Vector Random Variable}

\begin{figure}[h!]
\centerline{
\includegraphics[scale=0.40,clip=true]{figs/uqVectorRandomVariable.eps}
}
\caption{
{\color{red}{The class diagram for the vector random variable class}}.
}
\label{fig-vector-rv-class}
\end{figure}

\clearpage
\subsection{Statistical Inverse Problem (and Options)}

\begin{figure}[h!]
\centerline{
\includegraphics[scale=0.40,clip=true]{figs/uqSip.eps}
}
\caption{
The statistical inverse problem class. It implements the representation in Figure \ref{fig-sip-queso}.
}
\label{fig-sip-class}
\end{figure}

\begin{figure}[h!]
\begin{center}
\includegraphics[scale=0.40,clip=true]{figs/uqSipOptions.eps}
\end{center}
\caption{
The statistical inverse problem options class.
}
\label{fig-sip-options-class}
\end{figure}

\begin{table}[!h]
\begin{center}
\begin{tabular}{|l|c|c|}
\hline
\multicolumn{1}{|c|}{Option}                 & Default & Description \\
\multicolumn{1}{|c|}{Name}                   & Value   &             \\
\hline
\hline
$\langle$PREFIX$\rangle$help                 &         &             \\
\hline
$\langle$PREFIX$\rangle$computeSolution      &         &             \\
\hline
$\langle$PREFIX$\rangle$dataOutputFileName   &         &             \\
\hline
$\langle$PREFIX$\rangle$dataOutputAllowedSet &         &             \\
\hline
\end{tabular}
\end{center}
\caption{
Input file options for a QUESO statistical inverse problem.
}
\label{tab-sip-options}
\end{table}

\clearpage
\subsection{Markov Chain Solver (and Options)}

\begin{figure}[h!]
\centerline{
\includegraphics[scale=0.40,clip=true]{figs/uqMarkovChainSG.eps}
}
\caption{
The Markov chain sequence generator class.
}
\label{fig-markov-chain-solver-class}
\end{figure}

\begin{figure}[h!]
\begin{center}
\includegraphics[scale=0.40,clip=true]{figs/uqMarkovChainSGOptions.eps}
\end{center}
\caption{
The Markov chain sequence generator options class.
}
\label{fig-markov-options-class}
\end{figure}

\begin{table}[!h]
\begin{center}
\begin{tabular}{|l|c|c|}
\hline
\multicolumn{1}{|c|}{Option}                               & Default & Description \\
\multicolumn{1}{|c|}{Name}                                 & Value   &             \\
\hline
\hline
$\langle$PREFIX$\rangle$help                               &         &             \\
\hline
$\langle$PREFIX$\rangle$dataOutputFileName                 &         &             \\
\hline
$\langle$PREFIX$\rangle$dataOutputAllowedSet               &         &             \\
\hline
$\langle$PREFIX$\rangle$rawChain\_type                     &         &             \\
\hline
$\langle$PREFIX$\rangle$rawChain\_dataInputFileName         &         &             \\
\hline
$\langle$PREFIX$\rangle$rawChain\_size                      &         &             \\
\hline
$\langle$PREFIX$\rangle$rawChain\_generateExtra             &         &             \\
\hline
$\langle$PREFIX$\rangle$rawChain\_displayPeriod             &         &             \\
\hline
$\langle$PREFIX$\rangle$rawChain\_measureRunTimes           &         &             \\
\hline
$\langle$PREFIX$\rangle$rawChain\_dataOutputFileName        &         &             \\
\hline
$\langle$PREFIX$\rangle$rawChain\_dataOutputAllowedSet      &         &             \\
\hline
$\langle$PREFIX$\rangle$rawChain\_computeStats              &         &             \\
\hline
$\langle$PREFIX$\rangle$filteredChain\_generate             &         &             \\
\hline
$\langle$PREFIX$\rangle$filteredChain\_discardedPortion     &         &             \\
\hline
$\langle$PREFIX$\rangle$filteredChain\_lag                  &         &             \\
\hline
$\langle$PREFIX$\rangle$filteredChain\_dataOutputFileName   &         &             \\
\hline
$\langle$PREFIX$\rangle$filteredChain\_dataOutputAllowedSet &         &             \\
\hline
$\langle$PREFIX$\rangle$filteredChain\_computeStats         &         &             \\
\hline
$\langle$PREFIX$\rangle$mh\_displayCandidates               &         &             \\
\hline
$\langle$PREFIX$\rangle$mh\_putOutOfBoundsInChain           &         &             \\
\hline
$\langle$PREFIX$\rangle$tk\_useLocalHessian                 &         &             \\
\hline
$\langle$PREFIX$\rangle$tk\_useNewtonComponent              &         &             \\
\hline
$\langle$PREFIX$\rangle$dr\_maxNumExtraStages               &         &             \\
\hline
$\langle$PREFIX$\rangle$dr\_scalesForExtraStages            &         &             \\
\hline
$\langle$PREFIX$\rangle$am\_initialNonAdaptInterval         &         &             \\
\hline
$\langle$PREFIX$\rangle$am\_adaptInterval                   &         &             \\
\hline
$\langle$PREFIX$\rangle$am\_eta                             &         &             \\
\hline
$\langle$PREFIX$\rangle$am\_epsilon                         &         &             \\
\hline
\end{tabular}
\end{center}
\caption{
Input file options for a QUESO Markov chain solver.
}
\label{tab-markov-options}
\end{table}

\clearpage
\subsection{Statistical Forward Problem (and Options)}

\begin{figure}[h!]
\centerline{
\includegraphics[scale=0.40,clip=true]{figs/uqSfp.eps}
}
\caption{
The statistical forward problem class. It implements the representation in Figure \ref{fig-sfp-queso}.
}
\label{fig-sfp-class}
\end{figure}

\begin{figure}[h!]
\begin{center}
\includegraphics[scale=0.40,clip=true]{figs/uqSfpOptions.eps}
\end{center}
\caption{
The statistical forward problem options class.
}
\label{fig-sfp-options-class}
\end{figure}

\begin{table}[!h]
\begin{center}
\begin{tabular}{|l|c|c|}
\hline
\multicolumn{1}{|c|}{Option}                 & Default & Description \\
\multicolumn{1}{|c|}{Name}                   & Value   &             \\
\hline
\hline
$\langle$PREFIX$\rangle$help                 &         &             \\
\hline
$\langle$PREFIX$\rangle$computeSolution      &         &             \\
\hline
$\langle$PREFIX$\rangle$computeCovariances   &         &             \\
\hline
$\langle$PREFIX$\rangle$computeCorrelations  &         &             \\
\hline
$\langle$PREFIX$\rangle$dataOutputFileName   &         &             \\
\hline
$\langle$PREFIX$\rangle$dataOutputAllowedSet &         &             \\
\hline
\end{tabular}
\end{center}
\caption{
Input file options for a QUESO statistical forward problem.
}
\label{tab-sfp-options}
\end{table}

\clearpage
\subsection{Monte Carlo Solver (and Options)}

\begin{figure}[h!]
\centerline{
\includegraphics[scale=0.40,clip=true]{figs/uqMonteCarloSG.eps}
}
\caption{
{\color{red}{The Monte Carlo sequence generator class}}.
}
\label{fig-monte-carlo-solver-class}
\end{figure}

\begin{figure}[h!]
\begin{center}
\includegraphics[scale=0.40,clip=true]{figs/uqMonteCarloSGOptions.eps}
\end{center}
\caption{
{\color{red}{The Monte Carlo sequence generator options class}}.
}
\label{fig-monte-carlo-options-class}
\end{figure}

\begin{table}[!h]
\begin{center}
\begin{tabular}{|l|c|c|}
\hline
\multicolumn{1}{|c|}{Option}                        & Default & Description \\
\multicolumn{1}{|c|}{Name}                          & Value   &             \\
\hline
\hline
$\langle$PREFIX$\rangle$help                        &         &             \\
\hline
$\langle$PREFIX$\rangle$dataOutputFileName          &         &             \\
\hline
$\langle$PREFIX$\rangle$dataOutputAllowedSet        &         &             \\
\hline
$\langle$PREFIX$\rangle$pseq\_dataOutputFileName     &         &             \\
\hline
$\langle$PREFIX$\rangle$pseq\_dataOutputAllowedSet   &         &             \\
\hline
$\langle$PREFIX$\rangle$pseq\_computeStats           &         &             \\
\hline
$\langle$PREFIX$\rangle$qseq\_dataInputFileName      &         &             \\
\hline
$\langle$PREFIX$\rangle$qseq\_size                   &         &             \\
\hline
$\langle$PREFIX$\rangle$qseq\_displayPeriod          &         &             \\
\hline
$\langle$PREFIX$\rangle$qseq\_measureRunTimes        &         &             \\
\hline
$\langle$PREFIX$\rangle$qseq\_dataOutputFileName     &         &             \\
\hline
$\langle$PREFIX$\rangle$qseq\_dataOutputAllowedSet   &         &             \\
\hline
$\langle$PREFIX$\rangle$qseq\_computeStats           &         &             \\
\hline
\end{tabular}
\end{center}
\caption{
Input file options for a QUESO Monte Carlo solver.
}
\label{tab-monte-carlo-options}
\end{table}

\clearpage
\subsection{Options for Statistical Analysis of Sequences}

\begin{figure}[h!]
\centerline{
\includegraphics[scale=0.40,clip=true]{figs/uqSequenceStatisticalOptions.eps}
}
\caption{
{\color{red}{The sequence statistical options class}}.
}
\label{fig-seq-statistical-options-class}
\end{figure}

\begin{table}[!h]
\begin{center}
\begin{tabular}{|l|c|c|}
\hline
\multicolumn{1}{|c|}{Option}                      & Default & Description \\
\multicolumn{1}{|c|}{Name}                        & Value   &             \\
\hline
\hline
$\langle$PREFIX$\rangle$help                      &         &             \\
\hline
$\langle$PREFIX$\rangle$initialDiscardedPortions  &         &             \\
\hline
\hline
$\langle$PREFIX$\rangle$bmm\_run                   &         &             \\
\hline
$\langle$PREFIX$\rangle$bmm\_lengths               &         &             \\
\hline
$\langle$PREFIX$\rangle$bmm\_display               &         &             \\
\hline
$\langle$PREFIX$\rangle$bmm\_write                 &         &             \\
\hline
\hline
$\langle$PREFIX$\rangle$fft\_compute               &         &             \\
\hline
$\langle$PREFIX$\rangle$fft\_paramId               &         &             \\
\hline
$\langle$PREFIX$\rangle$fft\_size                  &         &             \\
\hline
$\langle$PREFIX$\rangle$fft\_testInversion         &         &             \\
\hline
$\langle$PREFIX$\rangle$fft\_write                 &         &             \\
\hline
\hline
$\langle$PREFIX$\rangle$psd\_compute               &         &             \\
\hline
$\langle$PREFIX$\rangle$psd\_numBlocks             &         &             \\
\hline
$\langle$PREFIX$\rangle$psd\_hopSizeRatio          &         &             \\
\hline
$\langle$PREFIX$\rangle$psd\_paramId               &         &             \\
\hline
$\langle$PREFIX$\rangle$psd\_write                 &         &             \\
\hline
\hline
$\langle$PREFIX$\rangle$psdAtZero\_compute         &         &             \\
\hline
$\langle$PREFIX$\rangle$psdAtZero\_numBlocks       &         &             \\
\hline
$\langle$PREFIX$\rangle$psdAtZero\_hopSizeRatio    &         &             \\
\hline
$\langle$PREFIX$\rangle$psdAtZero\_display         &         &             \\
\hline
$\langle$PREFIX$\rangle$psdAtZero\_write           &         &             \\
\hline
\hline
$\langle$PREFIX$\rangle$geweke\_compute            &         &             \\
\hline
$\langle$PREFIX$\rangle$geweke\_naRatio            &         &             \\
\hline
$\langle$PREFIX$\rangle$geweke\_nbRatio            &         &             \\
\hline
$\langle$PREFIX$\rangle$geweke\_display            &         &             \\
\hline
$\langle$PREFIX$\rangle$geweke\_write              &         &             \\
\hline
\hline
$\langle$PREFIX$\rangle$autoCorr\_computeViaDef    &         &             \\
\hline
$\langle$PREFIX$\rangle$autoCorr\_computeViaFft    &         &             \\
\hline
$\langle$PREFIX$\rangle$autoCorr\_secondLag        &         &             \\
\hline
$\langle$PREFIX$\rangle$autoCorr\_lagSpacing       &         &             \\
\hline
$\langle$PREFIX$\rangle$autoCorr\_numLags          &         &             \\
\hline
$\langle$PREFIX$\rangle$autoCorr\_display          &         &             \\
\hline
$\langle$PREFIX$\rangle$autoCorr\_write            &         &             \\
\hline
\hline
$\langle$PREFIX$\rangle$meanStacc\_compute         &         &             \\
\hline
\hline
$\langle$PREFIX$\rangle$hist\_compute              &         &             \\
\hline
$\langle$PREFIX$\rangle$hist\_numInternalBins      &         &             \\
\hline
\hline
$\langle$PREFIX$\rangle$cdfStacc\_compute          &         &             \\
\hline
$\langle$PREFIX$\rangle$cdfStacc\_numEvalPositions &         &             \\
\hline
\hline
$\langle$PREFIX$\rangle$kde\_compute               &         &             \\
\hline
$\langle$PREFIX$\rangle$kde\_numEvalPositions      &         &             \\
\hline
\hline
$\langle$PREFIX$\rangle$covMatrix\_compute         &         &             \\
\hline
$\langle$PREFIX$\rangle$corrMatrix\_compute        &         &             \\
\hline
\end{tabular}
\end{center}
\caption{
Input file options for a the statistical analysis of sequences.
}
\label{tab-seq-statistical-options}
\end{table}

