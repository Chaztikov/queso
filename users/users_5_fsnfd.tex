\chapter{Free Software Needs Free Documentation}\label{ch-fsnfd}
\thispagestyle{headings}
\markboth{Appendix \ref{ch-fsnfd}: Free Software Needs Free Documentation}{Appendix \ref{ch-fsnfd}: Free Software Needs Free Documentation}

% Taken from http://www.gnu.org/software/gsl/manual/html_node/Free-Software-Needs-Free-Documentation.html
%---------------------------------------------------------------------

%\begin{center}
{\it The following article was written by Richard Stallman, founder of the GNU Project.}
%\end{center}

The biggest deficiency in the free software community today is not in the software$-$it is the lack of good free documentation that we can include with the free software.
Many of our most important programs do not come with free reference manuals and free introductory texts. Documentation is an essential part of any software package; when
an important free software package does not come with a free manual and a free tutorial, that is a major gap. We have many such gaps today.

Consider Perl, for instance. The tutorial manuals that people normally use are non-free. How did this come about? Because the authors of those manuals published them
with restrictive terms$-$no copying, no modification, source files not available$-$which exclude them from the free software world.

That wasn't the first time this sort of thing happened, and it was far from the last. Many times we have heard a GNU user eagerly describe a manual that he is writing,
his intended contribution to the community, only to learn that he had ruined everything by signing a publication contract to make it non-free.

Free documentation, like free software, is a matter of freedom, not price. The problem with the non-free manual is not that publishers charge a price for printed copies$-$that
in itself is fine. (The Free Software Foundation sells printed copies of manuals, too.) The problem is the restrictions on the use of the manual. Free manuals are available
in source code form, and give you permission to copy and modify. Non-free manuals do not allow this.

The criteria of freedom for a free manual are roughly the same as for free software. Redistribution (including the normal kinds of commercial redistribution) must be
permitted, so that the manual can accompany every copy of the program, both on-line and on paper.

Permission for modification of the technical content is crucial too. When people modify the software, adding or changing features, if they are conscientious they will
change the manual too$-$so they can provide accurate and clear documentation for the modified program. A manual that leaves you no choice but to write a new manual to
document a changed version of the program is not really available to our community.

Some kinds of limits on the way modification is handled are acceptable. For example, requirements to preserve the original author's copyright notice, the distribution
terms, or the list of authors, are ok. It is also no problem to require modified versions to include notice that they were modified. Even entire sections that may not
be deleted or changed are acceptable, as long as they deal with nontechnical topics (like this one). These kinds of restrictions are acceptable because they don't
obstruct the community's normal use of the manual.

However, it must be possible to modify all the technical content of the manual, and then distribute the result in all the usual media, through all the usual channels.
Otherwise, the restrictions obstruct the use of the manual, it is not free, and we need another manual to replace it.

Please spread the word about this issue. Our community continues to lose manuals to proprietary publishing. If we spread the word that free software needs free reference
manuals and free tutorials, perhaps the next person who wants to contribute by writing documentation will realize, before it is too late, that only free manuals contribute
to the free software community.

If you are writing documentation, please insist on publishing it under the GNU Free Documentation License or another free documentation license. Remember that this decision
requires your approval$-$you don't have to let the publisher decide. Some commercial publishers will use a free license if you insist, but they will not propose the option;
it is up to you to raise the issue and say firmly that this is what you want. If the publisher you are dealing with refuses, please try other publishers. If you're not sure
whether a proposed license is free, write to lice

%---------------------------------------------------------------------

